\documentclass[conference]{IEEEtran}
\IEEEoverridecommandlockouts
% The preceding line is only needed to identify funding in the first footnote. If that is unneeded, please comment it out.
\usepackage{cite}
\usepackage{amsmath,amssymb,amsfonts}
\usepackage{algorithmic}
\usepackage{graphicx}
\usepackage{textcomp}
\usepackage{xcolor}
\def\BibTeX{{\rm B\kern-.05em{\sc i\kern-.025em b}\kern-.08em
    T\kern-.1667em\lower.7ex\hbox{E}\kern-.125emX}}
\begin{document}

\title{Kaugummi zur Übelkeitsminderung in der Virtual Reality\\}

\author{Philipp Lauer, Denis Schlusche, Marc Zintel}

\maketitle

\begin{abstract}
Abstract
\end{abstract}

\begin{IEEEkeywords}
component, formatting, style, styling, insert
\end{IEEEkeywords}

\section{Einleitung}
Viele Menschen, die schon einmal in die Virtual Reality eingetaucht sind, erleben dabei ein Gefühl der Benommenheit und Übelkeit. Diese Übelkeit nennt man Motion Sickness oder Virtual Reality Krankheit. Ein Ansatz, diese Motion Sickness zu "heilen", ist es für Ablenkung zu schaffen. Dies wird im Folgenden versucht, durch das Kauen eines Kaugummis zu schaffen. Dadurch soll der Kiefer in Bewegung gehalten werden und eine angenehme Ablenkung durch den Geschmack erzeugt werden. Es werden 20 Probanden in A/B-Tests gefordert, in einer Virtual Realtiy-Anwendung sechs Minuten zu verbringen und dabei stets neue Dinge zu entdecken. Die Anwendung ist ein Nachbau der US-Amerikanischen Großstadt San Francisco, weshalb es sinnvoll ist, markante Bauwerke als Ziel der Suche aufzugeben. 

\section{Idee}
Die grundsätzliche Idee hinter dem Versuch lautet: Durch das Kauen von Kaugummi, wird Menschen in der Virtual Reality weniger schlecht!
Diese Idee basiert darauf, dass durch das Kauen eine stetige Bewegung des Kiefers und Kopfes ausgelöst wird, welche das Gehirn zu einem gewissen Maß ablenkt. Dadurch soll das Auftreten von Motion Sickness verringert, beziehungsweise gemildert werden. 

\section{Durchführung}
Es werden 20 Probanden in einem A/B-Test in einer Virtual Reality-Anwendung zwei Versuchsdurchläufe absolvieren. 
\subsection{Versuchsaufbau}


\begin{thebibliography}{00}
\bibitem{b1} G. Eason, B. Noble, and I. N. Sneddon, ``On certain integrals of Lipschitz-Hankel type involving products of Bessel functions,'' Phil. Trans. Roy. Soc. London, vol. A247, pp. 529--551, April 1955.

\end{thebibliography}
\end{document}
